\documentclass[12pt,]{scrartcl}
\usepackage{multirow}
\usepackage{xcolor}
\usepackage{makecell}
\usepackage{style}
\date{}

\newenvironment{myitemize}
{ \begin{itemize}
    \setlength{\itemsep}{5pt}
    \setlength{\parskip}{0pt}
    \setlength{\parsep}{0pt}     }
{ \end{itemize}                  }

\begin{document}

\section{Research Articles}\label{publications}

\subsection{Published Articles}\label{articles}

\begin{myitemize}
\leftskip-0.25in % for bullets

\item \underline{L. Torres}, K. S. Chan, and T. Eliassi-Rad. \textbf{GLEE: Geometric Laplacian Eigenmap Embedding.} Journal of Complex Networks, Volume 8, Issue 2, April 2020, cnaa007. \href{https://academic.oup.com/comnet/article/8/2/cnaa007/5775302?guestAccessKey=a6a1e399-bc7d-48db-82ad-5a3beabd81bf}{[link]}

\item \underline{L. Torres}, P. Su\'arez-Serrato and T. Eliassi-Rad.  \textbf{Non-backtracking Cycles: Length Spectrum Theory and Graph Mining Applications.}  Appl Netw Sci (2019) 4: 41. \href{https://doi.org/10.1007/s41109-019-0147-y}{[link]}

\end{myitemize}


\subsection{Preprints}\label{preprints}

\begin{myitemize}
\leftskip-0.25in % for bullets

\item B. Klein, T. LaRock, S. McCabe, \underline{L. Torres}, L. Friedland, F. Privitera, B. Lake, M. U. G. Kraemer, J. S. Brownstein, D. Lazer, T. Eliassi-Rad, S. V. Scarpino, A. Vespignani, and M- Chinazzi. \textbf{Reshaping a nation: Mobility, commuting, and contact patterns during the COVID-19 outbreak.} Technical report (2020). \href{https://www.mobs-lab.org/uploads/6/7/8/7/6787877/covid19mobility_report2.pdf}{[link]}

\item B. Klein, T. LaRock, S. McCabe, \underline{L. Torres}, F. Privitera, B. Lake, M. U. G. Kraemer, J. S. Brownstein, D. Lazer, T. Eliassi-Rad, S. V. Scarpino, M- Chinazzi, and A. Vespignani. \textbf{Assessing changes in commuting and individual mobility in major metropolitan areas in the United States during the COVID-19 outbreak.} Technical report (2020). \href{https://www.mobs-lab.org/uploads/6/7/8/7/6787877/assessing_mobility_changes_in_the_united_states_during_the_covid_19_outbreak.pdf}{[link]}

\item \underline{L. Torres}, K. S. Chan, H. Tong and T. Eliassi-Rad. \textbf{Node Immunization with Non-backtracking Eigenvalues.} Preprint. arXiv:2002.12309 (2020). \href{https://arxiv.org/abs/2002.12309}{[link]}

\end{myitemize}


\section{Academic Presentations}\label{presentations}

\subsection{Invited Presentations}\label{invited-presentations}

\begin{myitemize}
\leftskip-0.25in % for bullets

\item \textbf{Non-Backtracking Cycles: Length Spectrum Theory and Graph Mining Applications.} \underline{L. Torres}, P. Su\'arez-Serrato, and T. Eliassi-Rad. MiDAS Research Group Seminar, Boston University. Boston, MA, USA. November 2019.

\end{myitemize}

\subsection{Contributed Presentations}\label{contributed-presentations}

\begin{myitemize}
\leftskip-0.25in % for bullets

\item \textbf{The Largest Non-Backtracking Eigenvalue under Node Removal.} \underline{L. Torres}, T. Eliassi-Rad. Student Research Symposium of the Network Science Institute. Boston, MA, USA. November 2019. \href{http://leotrs.com/static/symposium2019.pdf}{[slides]}

\item \textbf{GLEE: Geometric Laplacian Eigenmap Embedding.} \underline{L. Torres}, K. S. Chan, and T. Eliassi-Rad. Latin American Conference on Complex Networks (LANET'19). Cartagena, Colombia. August 2019. \href{http://leotrs.com/static/GLEE_lanet19.pdf}{[slides]}

\item \textbf{GLEE: Geometric Laplacian Eigenmap Embedding.} \underline{L. Torres}, K. S. Chan, and T. Eliassi-Rad. The 2019 International Conference on Network Science (NetSci'19). Burlington, VT, USA. May 2019. \href{http://leotrs.com/static/GLEE_netsci19.pdf}{[slides]}

\item \textbf{Graph Distance from a Topological View of Non-Backtracking Cycles.} \underline{L. Torres}, P. Suárez Serrato, T. Eliassi-Rad. Student Research Symposium of the Network Science Institute. Boston, MA, USA. November 2018. \href{http://leotrs.com/static/netsci18.pdf}{[slides]}

\item \textbf{A Bridge Between Homotopy Theory and Network Science.} \underline{L. Torres}, P. Suárez Serrato, T. Eliassi-Rad. SIAM Workshop on Network Science 2018 (SIAMNS'18). Portland, OR, USA. July 2018. \href{http://leotrs.com/static/siamns18.pdf}{[slides]}

\item \textbf{A Study of Cycle Length Spectra.} \underline{L. Torres}, P. Suárez Serrato, T. Eliassi-Rad. The 2018 International Conference on Network Science (NetSci'18). Paris, France. June 2018. \href{http://leotrs.com/static/netsci18.pdf}{[slides]}

\end{myitemize}

\subsection{Tutorials}\label{tutorials}

\begin{myitemize}
\leftskip-0.25in

\item Co-tutor for part 3 of Tutorial on \textbf{Graph Metric Spaces}. SIAM International Conference on Data Mining (SDM19), Calgary, Canada. May 2019. \url{https://neu-spiral.github.io/GraphMetricSpaces/}

\item Co-tutor for part 3 of Tutorial on \textbf{Graph Metric Spaces}. International Conference on Knowledge Discovery and Data Mining (KDD18), London, UK. August 2018. \url{https://neu-spiral.github.io/GraphMetricSpaces/}

\end{myitemize}


\subsection{Posters}\label{posters}

\begin{myitemize}
\leftskip-0.25in % for bullets

\item \textbf{The why, how, and when of representations for complex systems.} \underline{L. Torres} and A. Sizemore Blevins, D. S. Bassett and T. Eliassi-Rad. The 2019 International Conference on Network Science (NetSci'19). Burlington, Vermont, USA. May 2019. \href{http://leotrs.com/static/hyper_poster.pdf}{[poster]}

\item \textbf{GLEE: Geometric Laplacian Eigenmap Embedding.} \underline{L. Torres}, K. S. Chan, and T. Eliassi-Rad. New England Machine Learning Day 2019 (NEML'19). Boston, MA, USA. May 2019. \href{http://leotrs.com/static/glee_poster.pdf}{[poster]}

\item \textbf{GLEE: Geometric Laplacian Eigenmap Embedding.} \underline{L. Torres}, K. S. Chan, and T. Eliassi-Rad. Graph Exploitation Symposium (GraphEx'19). Dedham, MA, USA. April 2019. \href{http://leotrs.com/static/glee_poster.pdf}{[poster]}

\item \textbf{Graph Distance from the Topological Perspective of Nonbacktracking Cycles.} \underline{L. Torres} and T. Eliassi-Rad. New England Machine Learning Day 2018 (NEML'18). Cambridge, MA, USA. May 2018. \href{http://leotrs.com/static/neml18.pdf}{[poster]}

\item \textbf{A Bridge between Homotopy Theory and Network Science.} \underline{L. Torres} and T. Eliassi-Rad. Graph Exploitation Symposium (GraphEx'18). Dedham, MA, USA. April 2018. \href{http://leotrs.com/static/graphex18.pdf}{[poster]}

\item \textbf{A Study of Cycle Length Distributions: Asymptotics, Applications, and Links to Homotopy Theory.} \underline{L. Torres} and T. Eliassi-Rad. The 9th International Conference on Complex Networks (CompleNet'18). Boston, MA, USA. March 2018. \href{http://leotrs.com/static/complenet18.pdf}{[poster]}

\end{myitemize}



\section{Service}\label{academic-activities}

\subsection{Conferences and Symposia}\label{conferences}

\begin{myitemize}
\leftskip-0.25in

\item Co-organizer of the \textbf{Diversify NetSci} conference satellite. NetSci'20. September 2020. Rome, Italy.

\item Co-organizer of the \textbf{Diversify NetSci} conference satellite. NetSci'19. May 2019. Burlington, VT, USA. \url{https://www.networkscienceinstitute.org/diversifynetsci2019}

\item Co-organizer of the first \textbf{Student Research Symposium of NetSI}. Network Science Institute, Northeastern University. November 2018. Boston, MA, USA.

\item Co-organizer and lecturer of Linear Algebra at the \textbf{Network Science Institute Bootcamp for incoming PhD students}. August 2018. Boston, MA, USA.

\item Co-organizer of the \textbf{Society of Young Network Scientists} pre-conference event. CompleNet'18. March 2018. Boston, MA, USA.

\item Co-organizer of the first \textbf{Symposium for the Society of Young Network Scientists}. NetSci'17. June 2017. Indianapolis, IN, USA.

\item Co-organizer and lecturer of Linear Algebra at the first \textbf{Network Science Institute Bootcamp for incoming PhD students}. August 2017. Boston, MA, USA.

\end{myitemize}


\subsection{Journal Referee}\label{journal-referee}

\begin{myitemize}
\leftskip-0.25in

\item IEEE Transactions on Knowledge and Data Engineering (TKDE).

\item Journal of Machine Learning Research (JMLR).

\item Proceedings of the Royal Society A (Proceedings A).

\item Random Matrices: Theory and Applications (RMTA)

\end{myitemize}


\subsection{Published Software (non peer-reviewed)}\label{software}
\begin{myitemize}
\leftskip-0.25in

\item \textbf{netrd} \href{https://github.com/netsiphd/netrd}{[link]} netrd is a multi-purpose library with dozens of state-of-the-art implementations of algorithms for simulating dynamics on networks, measuring the distance between networks, and reconstructing networks from temporal data.

\item \textbf{sunbeam} \href{https://github.com/leotrs/sunbeam}{[link]} sunbeam is a library that uses the non-backtracking matrix to provide functionality for graph mining such as graph distance and graph embedding.

\item \textbf{glee} \href{https://github.com/leotrs/glee}{[link]} glee is a library that uses the simplex geometry of the Laplacian matrix to compute a geometric embedidng of an undirected graph.

\item \textbf{decu} \href{https://github.com/leotrs/decu}{[link]} decu is a suite of command line tools to automate the menial tasks involved in the development of experimental computation projects.

\item \textbf{erdos} \href{http://www.erdosnet.work}{[link]} erdos is an educational site for learning about and practicing Network Science through programming exercises.

\end{myitemize}

\end{document}
