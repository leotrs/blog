\documentclass[12pt,]{scrartcl}
\usepackage{multirow}
\usepackage{xcolor}
\usepackage{makecell}
\usepackage{style}
\date{}

%\defaultfontfeatures{Scale=MatchLowercase,Ligatures={TeX}}
%\setmainfont{Liberation Serif}
%\setsansfont{Liberation Serif}
%\setmonofont[SmallCapsFont={Liberation Serif}]{Liberation Serif}

\begin{document}

\section{Education}\label{education}
\textbf{\textsuperscript{†}} \emph{Indicates expected}

\begin{table}[h]
{\def\arraystretch{1.5}\tabcolsep=0pt
\begin{tabular}{p{0.18\linewidth}p{0.8\linewidth}}

2016--2021 \textsuperscript{\textdagger} & \textbf{Ph.D. Network Science, Northeastern University} \\

& Focus: Geometric Aspects of Complex Networks and Graph Mining \\
& Supervisor: Tina Eliassi-Rad \\
& Interests: network science, complexity science, applied mathematics
(probability, spectral linear algebra, algebraic topology, differential geometry), data science (graph mining, theoretical foundations, unsupervised machine learning), topological and geometric data analysis, design and development of algorithms, philosophy of network science \\
& Boston, MA, USA \\

2009--2015 & \textbf{B.Sc. Mathematics, \textit{Pontificia Universidad Católica del Perú}} \\
& Lima, Perú \\

2013--2014 & \textbf{College of The Holy Cross} \\
& Study abroad \& Spanish teaching assistant \\
& Worcester, MA \\
\end{tabular}%
}
\end{table}


\section{Selected Honours and
Awards}\label{selected-honours-and-awards}

\begin{table}[!h]
{\def\arraystretch{1.5}\tabcolsep=0pt
\begin{tabular}{p{0.10\linewidth}p{0.85\linewidth}}

2019 & Network Science Institute Travel Grant \\
& Financial aid to attend the 2019 Latin American Conference on Complex Networks \\

2015 & \textit{Pontificia Universidad Católica del Perú} \\
& Grades within top 3\% in the 75-year history of the Sciences and Engineering Department \\

\end{tabular}%
}
\end{table}


\newpage
\section{Research Articles and Presentations}\label{publications}

\subsection{Preprints}\label{preprints}

\begin{itemize}
\leftskip-0.25in % for bullets

\item Leo Torres, K. S. Chan and T. Eliassi-Rad. \textbf{Geometric Laplacian Eigenmap Embedding.} Preprint, arXiv:1905.09763 [cs.LG], 2019. Under review. \href{https://arxiv.org/abs/1905.09763}{[preprint]}

\item Leo Torres, P. Suárez Serrato and T. Eliassi-Rad. \textbf{Non-backtracking Cycles: Length Spectrum Theory and Graph Mining Applications.} In press, Applied Network Science 2019. \href{https://arxiv.org/abs/1807.09592}{[preprint]}

\end{itemize}


\subsection{Contributed Presentations}\label{contributed-presentations}

\begin{itemize}
\leftskip-0.25in

\item \textbf{GLEE: Geometric Laplacian Eigenmap Embedding.} Leo Torres, K. S. Chan, and T. Eliassi-Rad. The 2019 International Conference on Network Science (NetSci'19). Burlington, VT, USA. May 2019. \href{http://leotrs.com/static/GLEE_netsci19.pdf}{[slides]}

\item \textbf{Graph Distance from a Topological View of Non-Backtracking Cycles.} Leo Torres, P. Suárez Serrato, T. Eliassi-Rad. Student Research Symposium of the Network Science Institute. Boston, Massachusetts, USA. November 2018. \href{http://leotrs.com/static/netsci18.pdf}{[slides]}

\item \textbf{A Bridge Between Homotopy Theory and Network Science.} \textbf{Leo Torres}, P. Suárez Serrato, T. Eliassi-Rad. SIAM Workshop on Network Science 2018 (SIAMNS'18). Portland, Oregon, USA. July 2018. \href{http://leotrs.com/static/siamns18.pdf}{[slides]}

\item \textbf{A Study of Cycle Length Spectra.} \textbf{Leo Torres}, P. Suárez Serrato, T. Eliassi-Rad. The 2018 International Conference on Network Science (NetSci'18). Paris, France. June 2018. \href{http://leotrs.com/static/netsci18.pdf}{[slides]}

\end{itemize}


\subsection{Posters}\label{posters}

\begin{itemize}
\leftskip-0.25in % for bullets

\item \textbf{The why, how, and when of representations for complex systems.} Leo Torres and Ann Sizemore Blevins, Danielle S. Bassett and Tina Eliassi-Rad. The 2019 International Conference on Network Science (NetSci'19). Burlington, Vermont, USA. May 2019. \href{http://leotrs.com/static/hyper_poster.pdf}{[poster]}

\item \textbf{GLEE: Geometric Laplacian Eigenmap Embedding.} Leo Torres, K. S. Chan, and T. Eliassi-Rad. New England Machine Learning Day 2019 (NEML'19). Boston, MA, USA. May 2019. \href{http://leotrs.com/static/glee_poster.pdf}{[poster]}

\item \textbf{GLEE: Geometric Laplacian Eigenmap Embedding.} Leo Torres, K. S. Chan, and T. Eliassi-Rad. Graph Exploitation Symposium (GraphEx'19). Dedham, MA, USA. April 2019. \href{http://leotrs.com/static/glee_poster.pdf}{[poster]}

\item \textbf{Graph Distance from the Topological Perspective of Nonbacktracking Cycles.} Leo Torres and T. Eliassi-Rad. New England Machine Learning Day 2018 (NEML'18). Cambridge, MA, USA. May 2018. \href{http://leotrs.com/static/neml18.pdf}{[poster]}

\item \textbf{A Bridge between Homotopy Theory and Network Science.} Leo Torres and T. Eliassi-Rad. Graph Exploitation Symposium (GraphEx'18). Dedham, MA, USA. April 2018. \href{http://leotrs.com/static/graphex18.pdf}{[poster]}

\item \textbf{A Study of Cycle Length Distributions: Asymptotics, Applications, and Links to Homotopy Theory.} Leo Torres and T. Eliassi-Rad. The 9th International Conference on Complex Networks (CompleNet'18). Boston, MA, USA. March 2018. \href{http://leotrs.com/static/complenet18.pdf}{[poster]}

\end{itemize}


\subsection{Tutorials}\label{tutorials}

\begin{itemize}
\leftskip-0.25in

\item Co-tutor for part 3 of Tutorial on \textbf{Graph Metric Spaces}. SIAM International Conference on Data Mining (SDM19), Calgary, Canada. May 2019. \url{https://neu-spiral.github.io/GraphMetricSpaces/}

\item Co-tutor for part 3 of Tutorial on \textbf{Graph Metric Spaces}. International Conference on Knowledge Discovery and Data Mining (KDD18), London, UK. August 2018. \url{https://neu-spiral.github.io/GraphMetricSpaces/}

\end{itemize}


\section{Other Academic Activities}\label{academic-activities}

\subsection{Conferences and Symposia}\label{conferences}

\begin{itemize}
\leftskip-0.25in

\item Co-organizer of the \textbf{Diversify NetSci} conference satellite. NetSci'19. May 2019. Burlington, VT, USA. \url{https://www.networkscienceinstitute.org/diversifynetsci2019}

\item Co-organizer of the first \textbf{Student Research Symposium of NetSI}. Network Science Institute, Northeastern University. November 2018. Boston, MA, USA.

\item Co-organizer and lecturer of Linear Algebra at the \textbf{Network Science Institute Bootcamp for incoming PhD students}. August 2018. Boston, MA, USA.

\item Co-organizer of the \textbf{Society of Young Network Scientists} pre-conference event. CompleNet'18. March 2018. Boston, MA, USA.

\item Co-organizer of the first \textbf{Symposium for the Society of Young Network Scientists}. NetSci'17. June 2017. Indianapolis, IN, USA.

\item Co-organizer and lecturer of Linear Algebra at the first \textbf{Network Science Institute Bootcamp for incoming PhD students}. August 2017. Boston, MA, USA.

\end{itemize}


\subsection{Software}\label{software}
\begin{itemize}
\leftskip-0.25in

\item \textbf{netrd} [\url{github.com/netsiphd/netrd}] netrd is a multi-purpose library with dozens of state-of-the-art implementations of algorithms for simulating dynamics on networks, measuring the distance between networks, and reconstructing networks from temporal data.

\item \textbf{sunbeam} [\url{github.com/leotrs/sunbeam}] sunbeam is a library that uses the non-backtracking matrix to provide functionality for graph mining such as graph distance and graph embedding.

\item \textbf{decu} [\url{github.com/leotrs/decu}] decu is a suite of command line tools to automate the menial tasks involved in the development of experimental computation projects.

\item \textbf{erdos} [\url{www.erdosnet.work}] erdos is an educational site for learning about and practicing Network Science through programming exercises.

\end{itemize}


\newpage


\section{Professional Experience}\label{experience}
\begin{table}[h]
{\def\arraystretch{1.5}\tabcolsep=0pt
\begin{tabular}{p{0.18\linewidth}p{0.8\linewidth}}

Summer 2019 & Intern -- \textbf{Yahoo! Research} \\
& Graph mining and machine learning intern under the supervision of Yifan Hu. \\
& New York, NY, USA \\

Spring 2016 & Attendant -- \textbf{Recurse Center} \\
& Spent twelve weeks at a programmers' retreat focusing full-time on developing programming skills in a self-directed way. Focus on algorithm design and high-quality code writing standards. \\
& New York, NY, USA \\

Spring 2015 & \makecell[l]{\textbf{Calculus 4} Teaching Assistant -- \textit{Pontificia Universidad} \\ \textit{Católica del Perú}} \\
& Proctored and graded exams. \\
& Lima, Perú \\

Spring 2015 & \makecell[l]{\textbf{Organizer of the first Real Analysis Summer School} -- \\ \textit{Pontificia Universidad Católica del Perú}} \\
& Taught real analysis at the undergraduate level, designed and graded homework, gave lectures, supervised presentations. \\
& Lima, Perú \\

2013 -- 2014 & \textbf{Foreing Language Assistant} -- College of The Holy Cross \\ & Directed Spanish conversation lessons focusing on speaking, listening, and cultural sharing. Basic, intermadiate, and advanced levels. \\
& Worcester, MA, USA \\

2012 - 2014 & \textbf{Research Programmer} -- Wolfram Research South America \\
& Content development for the Wolfram|Alpha knowledge engine. \\
& Lima, Perú\\

\end{tabular}
}
\end{table}


\section{Miscellaneous}\label{misc}

\begin{itemize}
\leftskip-0.25in

\item Languages: Spanish (native), English (bilingual), French (beginner).

\item Computer skills: Python (expert), Mathematica, Linux, LaTeX (advanced), MATLAB, C/C++, R, Javascript, lua, LISP, Haskell (intermediate).

\item Advocacy: Open \{Science, Source, Data\}, inclusion and diversity.

\end{itemize}

\end{document}
