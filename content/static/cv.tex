\documentclass[12pt,]{scrartcl}
\usepackage{multirow}
\usepackage{xcolor}
\usepackage{makecell}
\usepackage{style}
\date{}

\begin{document}

\begin{table}[h]
{
\def\arraystretch{1.2}\tabcolsep=0pt
\begin{tabular}{p{0.60\linewidth}p{0.05\linewidth}p{0.35\linewidth}}

\multirow{1}{*}{\LARGE \textbf{Leonardo Torres}} & & \\
& & \\

Network Science Institute & \multicolumn{1}{r}{Email:\;\;} & \href{mailto:leo@leotrs.com}{leo@leotrs.com} \\

Northeastern University & \multicolumn{1}{r}{Home:\;\;} & \href{http://leo@leotrs.com}{leotrs.com} \\

Boston 02115, MA, USA & \multicolumn{1}{r}{Code:\;\;} & \href{https://github.com/leotrs}{github.com/leotrs} \\

\end{tabular}}
\end{table}

\section{Education}\label{education}
\vspace{-10px}
\textbf{\textsuperscript{†}} \emph{Indicates expected}

\begin{table}[h]
{\def\arraystretch{1}\tabcolsep=0pt
\begin{tabular}{p{0.15\linewidth}p{0.8\linewidth}}

2016--2021\textsuperscript{\textdagger} & \textbf{Ph.D. Network Science, Northeastern University} \\

& Focus: \textit{Geometric Aspects of Mining Complex Networks} \\
& Advisor: \textit{Tina Eliassi-Rad} \\
& Dissertation Proposal: \textit{November 2019} \\
& Dissertation Committee: \textit{Rose Yu, Dmitri Krioukov, Cristopher Moore, Tina Eliassi-Rad (Chair)} \\
& Boston, MA, USA \\ \\

2009--2015 & \textbf{B.Sc. Mathematics, \textit{Pontificia Universidad Católica del Perú}} \\
& Lima, Perú \\ \\

2013--2014 & \textbf{College of The Holy Cross} \\
& Study abroad \& Spanish teaching assistant \\
& Worcester, MA \\ \\
\end{tabular}}
\end{table}


\section{Selected Honours and
Awards}\label{selected-honours-and-awards}

\begin{table}[!h]
{\def\arraystretch{1}\tabcolsep=0pt
\begin{tabular}{p{0.2\linewidth}p{0.8\linewidth}}

July 2019 & LANET'19 Scholarship for young researchers \\
& Financial aid for attendance to the LANET'19 conference \\ \\

May 2019 & Network Science Institute Travel Grant \\
& Financial aid for academic travel in Summer 2019 \\ \\

2015 & \textit{Pontificia Universidad Católica del Perú} \\
& Grades within top 3\% in the 75-year history of the Sciences and Engineering Department \\ \\

\end{tabular}}
\end{table}


\newpage
\section{Research Articles}\label{publications}

\subsection{Published Articles}\label{articles}

\begin{itemize}
\leftskip-0.25in % for bullets

\item Leo Torres, P. Su\'arez-Serrato and T. Eliassi-Rad.  \textbf{Non-backtracking cycles: length spectrum theory and graph mining applications.}  Appl Netw Sci (2019) 4: 41. \href{https://doi.org/10.1007/s41109-019-0147-y}{[link]}

\item Leo Torres, K. S. Chan and T. Eliassi-Rad. \textbf{GLEE: Geometric Laplacian Eigenmap Embedding.} J. Complex Netw. In press, 2020. \href{https://arxiv.org/abs/1905.09763}{[preprint link]}

\end{itemize}


\subsection{Preprints}\label{preprints}

\begin{itemize}
\leftskip-0.25in % for bullets

\item Leo Torres, K. S. Chan, H. Tong and T. Eliassi-Rad. \textbf{Node Immunization with Non-backtracking Eigenvalues.} Preprint. arXiv:2002.12309 (2020) \href{https://arxiv.org/abs/2002.12309}{[link]}

\end{itemize}


\section{Academic Presentations}\label{presentations}

\subsection{Invited Presentations}\label{invited-presentations}

\begin{itemize}
\leftskip-0.25in % for bullets

\item \textbf{Non-Backtracking Cycles: Length Spectrum Theory and Graph Mining Applications.} Torres, L., Su\'arez-Serrato, P. and Eliassi-Rad, T. MiDAS Research Group Seminar, Boston University. Boston, MA, USA. November 2019.

\end{itemize}

\subsection{Contributed Presentations}\label{contributed-presentations}

\begin{itemize}
\leftskip-0.25in % for bullets

\item \textbf{The Largest Non-Backtracking Eigenvalue under Node Removal.} Leo Torres, T. Eliassi-Rad. Student Research Symposium of the Network Science Institute. Boston, MA, USA. November 2019. \href{http://leotrs.com/static/symposium2019.pdf}{[slides]}

\item \textbf{GLEE: Geometric Laplacian Eigenmap Embedding.} Leo Torres, K. S. Chan, and T. Eliassi-Rad. Latin American Conference on Complex Networks (LANET'19). Cartagena, Colombia. August 2019. \href{http://leotrs.com/static/GLEE_lanet19.pdf}{[slides]}

\item \textbf{GLEE: Geometric Laplacian Eigenmap Embedding.} Leo Torres, K. S. Chan, and T. Eliassi-Rad. The 2019 International Conference on Network Science (NetSci'19). Burlington, VT, USA. May 2019. \href{http://leotrs.com/static/GLEE_netsci19.pdf}{[slides]}

\item \textbf{Graph Distance from a Topological View of Non-Backtracking Cycles.} Leo Torres, P. Suárez Serrato, T. Eliassi-Rad. Student Research Symposium of the Network Science Institute. Boston, MA, USA. November 2018. \href{http://leotrs.com/static/netsci18.pdf}{[slides]}

\item \textbf{A Bridge Between Homotopy Theory and Network Science.} Leo Torres, P. Suárez Serrato, T. Eliassi-Rad. SIAM Workshop on Network Science 2018 (SIAMNS'18). Portland, OR, USA. July 2018. \href{http://leotrs.com/static/siamns18.pdf}{[slides]}

\item \textbf{A Study of Cycle Length Spectra.} Leo Torres, P. Suárez Serrato, T. Eliassi-Rad. The 2018 International Conference on Network Science (NetSci'18). Paris, France. June 2018. \href{http://leotrs.com/static/netsci18.pdf}{[slides]}

\end{itemize}

\subsection{Tutorials}\label{tutorials}

\begin{itemize}
\leftskip-0.25in

\item Co-tutor for part 3 of Tutorial on \textbf{Graph Metric Spaces}. SIAM International Conference on Data Mining (SDM19), Calgary, Canada. May 2019. \url{https://neu-spiral.github.io/GraphMetricSpaces/}

\item Co-tutor for part 3 of Tutorial on \textbf{Graph Metric Spaces}. International Conference on Knowledge Discovery and Data Mining (KDD18), London, UK. August 2018. \url{https://neu-spiral.github.io/GraphMetricSpaces/}

\end{itemize}


\subsection{Posters}\label{posters}

\begin{itemize}
\leftskip-0.25in % for bullets

\item \textbf{The why, how, and when of representations for complex systems.} Leo Torres and Ann Sizemore Blevins, Danielle S. Bassett and Tina Eliassi-Rad. The 2019 International Conference on Network Science (NetSci'19). Burlington, Vermont, USA. May 2019. \href{http://leotrs.com/static/hyper_poster.pdf}{[poster]}

\item \textbf{GLEE: Geometric Laplacian Eigenmap Embedding.} Leo Torres, K. S. Chan, and T. Eliassi-Rad. New England Machine Learning Day 2019 (NEML'19). Boston, MA, USA. May 2019. \href{http://leotrs.com/static/glee_poster.pdf}{[poster]}

\item \textbf{GLEE: Geometric Laplacian Eigenmap Embedding.} Leo Torres, K. S. Chan, and T. Eliassi-Rad. Graph Exploitation Symposium (GraphEx'19). Dedham, MA, USA. April 2019. \href{http://leotrs.com/static/glee_poster.pdf}{[poster]}

\item \textbf{Graph Distance from the Topological Perspective of Nonbacktracking Cycles.} Leo Torres and T. Eliassi-Rad. New England Machine Learning Day 2018 (NEML'18). Cambridge, MA, USA. May 2018. \href{http://leotrs.com/static/neml18.pdf}{[poster]}

\item \textbf{A Bridge between Homotopy Theory and Network Science.} Leo Torres and T. Eliassi-Rad. Graph Exploitation Symposium (GraphEx'18). Dedham, MA, USA. April 2018. \href{http://leotrs.com/static/graphex18.pdf}{[poster]}

\item \textbf{A Study of Cycle Length Distributions: Asymptotics, Applications, and Links to Homotopy Theory.} Leo Torres and T. Eliassi-Rad. The 9th International Conference on Complex Networks (CompleNet'18). Boston, MA, USA. March 2018. \href{http://leotrs.com/static/complenet18.pdf}{[poster]}

\end{itemize}



\section{Other Academic Activities}\label{academic-activities}

\subsection{Conferences and Symposia}\label{conferences}

\begin{itemize}
\leftskip-0.25in

\item Co-organizer of the \textbf{Diversify NetSci} conference satellite. NetSci'20. June 2020. Rome, Italy.

\item Co-organizer of the \textbf{Diversify NetSci} conference satellite. NetSci'19. May 2019. Burlington, VT, USA. \url{https://www.networkscienceinstitute.org/diversifynetsci2019}

\item Co-organizer of the first \textbf{Student Research Symposium of NetSI}. Network Science Institute, Northeastern University. November 2018. Boston, MA, USA.

\item Co-organizer and lecturer of Linear Algebra at the \textbf{Network Science Institute Bootcamp for incoming PhD students}. August 2018. Boston, MA, USA.

\item Co-organizer of the \textbf{Society of Young Network Scientists} pre-conference event. CompleNet'18. March 2018. Boston, MA, USA.

\item Co-organizer of the first \textbf{Symposium for the Society of Young Network Scientists}. NetSci'17. June 2017. Indianapolis, IN, USA.

\item Co-organizer and lecturer of Linear Algebra at the first \textbf{Network Science Institute Bootcamp for incoming PhD students}. August 2017. Boston, MA, USA.

\end{itemize}


\subsection{Journal Referee}\label{journal-referee}

\begin{itemize}
\leftskip-0.25in

\item IEEE Transactions on Knowledge and Data Engineering (TKDE).

\item Journal of Machine Learning Research (JMLR).

\item Proceedings of the Royal Society A (Proceedings A).

\end{itemize}


\subsection{Published Software (non peer-reviewed)}\label{software}
\begin{itemize}
\leftskip-0.25in

\item \textbf{netrd} \href{https://github.com/netsiphd/netrd}{[link]} netrd is a multi-purpose library with dozens of state-of-the-art implementations of algorithms for simulating dynamics on networks, measuring the distance between networks, and reconstructing networks from temporal data.

\item \textbf{sunbeam} \href{https://github.com/leotrs/sunbeam}{[link]} sunbeam is a library that uses the non-backtracking matrix to provide functionality for graph mining such as graph distance and graph embedding.

\item \textbf{glee} \href{https://github.com/leotrs/glee}{[link]} glee is a library that uses the simplex geometry of the Laplacian matrix to compute a geometric embedidng of an undirected graph.

\item \textbf{decu} \href{https://github.com/leotrs/decu}{[link]} decu is a suite of command line tools to automate the menial tasks involved in the development of experimental computation projects.

\item \textbf{erdos} \href{http://www.erdosnet.work}{[link]} erdos is an educational site for learning about and practicing Network Science through programming exercises.

\end{itemize}


\newpage


\section{Professional Experience}\label{experience}
\begin{table}[h]
{\def\arraystretch{1}\tabcolsep=0pt
\begin{tabular}{p{0.18\linewidth}p{0.8\linewidth}}

Summer 2019 &
Research Intern -- \textbf{Yahoo! Research} (New York, NY, USA) \\
& Machine learning intern under the supervision of Yifan Hu. \\ \\

Spring 2016 &
Attendant -- \textbf{Recurse Center} (New York, NY, USA) \\
& Spent twelve weeks at a programmers' retreat focusing full-time on developing programming skills in a self-directed way. Focus on algorithm design and high-quality code writing standards. \\ \\

Spring 2015 &
\textbf{Calculus Teaching Assistant} -- \textit{Pontificia Universidad Católica del Perú} (Lima, Perú) \\
& Proctored and graded exams. \\ \\

Spring 2015 &
\textbf{First Real Analysis Summer School} -- \textit{Pontificia Universidad Católica del Perú} (Lima, Perú) \\
& Main organizer; taught real analysis at the undergraduate level, designed and graded homework, gave lectures, supervised presentations. \\ \\

2013 -- 2014 &
\textbf{Foreing Language Assistant} -- College of The Holy Cross (Worcester, MA, USA) \\
& Directed Spanish conversation lessons focusing on speaking, listening, and cultural sharing. Basic, intermadiate, and advanced levels. \\ \\

2012 - 2014 &
\textbf{Research Programmer} -- Wolfram Research South America (Lima, Perú) \\
& Content development for the Wolfram|Alpha knowledge engine. \\

\end{tabular}}
\end{table}


\section{Miscellaneous}\label{misc}

\begin{itemize}
\leftskip-0.25in

\item Languages: Spanish (native), English (bilingual), French (beginner).

\item Computer skills: Python (expert), Mathematica, Linux, LaTeX (advanced), MATLAB, C/C++, R, Javascript, lua, LISP (intermediate).

\item Advocacy: Open \{Science, Source, Data\}, inclusion and diversity.

\end{itemize}

\end{document}
