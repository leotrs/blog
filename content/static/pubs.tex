\documentclass[12pt,]{scrartcl}
\usepackage{multirow}
\usepackage{xcolor}
\usepackage{makecell}
\usepackage{style}
\date{}

%\defaultfontfeatures{Scale=MatchLowercase,Ligatures={TeX}}
%\setmainfont{Liberation Serif}
%\setsansfont{Liberation Serif}
%\setmonofont[SmallCapsFont={Liberation Serif}]{Liberation Serif}

\begin{document}

\section{Research Articles}\label{publications}

\subsection{Published Articles}\label{articles}

\begin{itemize}
\leftskip-0.25in % for bullets

\item Torres, L., Su\'arez-Serrato, P. and Eliassi-Rad, T.  \textbf{Non-backtracking cycles: length spectrum theory and graph mining applications.}  Appl Netw Sci (2019) 4: 41. \href{https://doi.org/10.1007/s41109-019-0147-y}{[link]}

\end{itemize}


\subsection{Preprints}\label{preprints}

\begin{itemize}
\leftskip-0.25in % for bullets

\item Leo Torres, K. S. Chan and T. Eliassi-Rad. \textbf{Geometric Laplacian Eigenmap Embedding.} Preprint, arXiv:1905.09763 [cs.LG], 2019. Under review. \href{https://arxiv.org/abs/1905.09763}{[preprint]}

\end{itemize}


\section{Academic Presentations}\label{presentations}

\subsection{Invited Presentations}\label{invited-presentations}

\begin{itemize}
\leftskip-0.25in % for bullets

\item \textbf{Non-Backtracking Cycles: Length Spectrum Theory and Graph Mining Applications.} Torres, L., Su\'arez-Serrato, P. and Eliassi-Rad, T. MiDAS Research Group Seminar, Boston University. Boston, MA, USA. November 2019.

\end{itemize}

\subsection{Contributed Presentations}\label{contributed-presentations}

\begin{itemize}
\leftskip-0.25in % for bullets

\item \textbf{The Largest Non-Backtracking Eigenvalue under Node Removal.} Leo Torres, T. Eliassi-Rad. Student Research Symposium of the Network Science Institute. Boston, MA, USA. November 2019. \href{http://leotrs.com/static/symposium2019.pdf}{[slides]}

\item \textbf{GLEE: Geometric Laplacian Eigenmap Embedding.} Leo Torres, K. S. Chan, and T. Eliassi-Rad. Latin American Conference on Complex Networks (LANET'19). Cartagena, Colombia. August 2019. \href{http://leotrs.com/static/GLEE_lanet19.pdf}{[slides]}

\item \textbf{GLEE: Geometric Laplacian Eigenmap Embedding.} Leo Torres, K. S. Chan, and T. Eliassi-Rad. The 2019 International Conference on Network Science (NetSci'19). Burlington, VT, USA. May 2019. \href{http://leotrs.com/static/GLEE_netsci19.pdf}{[slides]}

\item \textbf{Graph Distance from a Topological View of Non-Backtracking Cycles.} Leo Torres, P. Suárez Serrato, T. Eliassi-Rad. Student Research Symposium of the Network Science Institute. Boston, MA, USA. November 2018. \href{http://leotrs.com/static/netsci18.pdf}{[slides]}

\item \textbf{A Bridge Between Homotopy Theory and Network Science.} Leo Torres, P. Suárez Serrato, T. Eliassi-Rad. SIAM Workshop on Network Science 2018 (SIAMNS'18). Portland, OR, USA. July 2018. \href{http://leotrs.com/static/siamns18.pdf}{[slides]}

\item \textbf{A Study of Cycle Length Spectra.} Leo Torres, P. Suárez Serrato, T. Eliassi-Rad. The 2018 International Conference on Network Science (NetSci'18). Paris, France. June 2018. \href{http://leotrs.com/static/netsci18.pdf}{[slides]}

\end{itemize}

\subsection{Tutorials}\label{tutorials}

\begin{itemize}
\leftskip-0.25in

\item Co-tutor for part 3 of Tutorial on \textbf{Graph Metric Spaces}. SIAM International Conference on Data Mining (SDM19), Calgary, Canada. May 2019. \url{https://neu-spiral.github.io/GraphMetricSpaces/}

\item Co-tutor for part 3 of Tutorial on \textbf{Graph Metric Spaces}. International Conference on Knowledge Discovery and Data Mining (KDD18), London, UK. August 2018. \url{https://neu-spiral.github.io/GraphMetricSpaces/}

\end{itemize}


\subsection{Posters}\label{posters}

\begin{itemize}
\leftskip-0.25in % for bullets

\item \textbf{The why, how, and when of representations for complex systems.} Leo Torres and Ann Sizemore Blevins, Danielle S. Bassett and Tina Eliassi-Rad. The 2019 International Conference on Network Science (NetSci'19). Burlington, Vermont, USA. May 2019. \href{http://leotrs.com/static/hyper_poster.pdf}{[poster]}

\item \textbf{GLEE: Geometric Laplacian Eigenmap Embedding.} Leo Torres, K. S. Chan, and T. Eliassi-Rad. New England Machine Learning Day 2019 (NEML'19). Boston, MA, USA. May 2019. \href{http://leotrs.com/static/glee_poster.pdf}{[poster]}

\item \textbf{GLEE: Geometric Laplacian Eigenmap Embedding.} Leo Torres, K. S. Chan, and T. Eliassi-Rad. Graph Exploitation Symposium (GraphEx'19). Dedham, MA, USA. April 2019. \href{http://leotrs.com/static/glee_poster.pdf}{[poster]}

\item \textbf{Graph Distance from the Topological Perspective of Nonbacktracking Cycles.} Leo Torres and T. Eliassi-Rad. New England Machine Learning Day 2018 (NEML'18). Cambridge, MA, USA. May 2018. \href{http://leotrs.com/static/neml18.pdf}{[poster]}

\item \textbf{A Bridge between Homotopy Theory and Network Science.} Leo Torres and T. Eliassi-Rad. Graph Exploitation Symposium (GraphEx'18). Dedham, MA, USA. April 2018. \href{http://leotrs.com/static/graphex18.pdf}{[poster]}

\item \textbf{A Study of Cycle Length Distributions: Asymptotics, Applications, and Links to Homotopy Theory.} Leo Torres and T. Eliassi-Rad. The 9th International Conference on Complex Networks (CompleNet'18). Boston, MA, USA. March 2018. \href{http://leotrs.com/static/complenet18.pdf}{[poster]}

\end{itemize}


\subsection{Published Software (non peer-reviewed)}\label{software}
\begin{itemize}
\leftskip-0.25in

\item \textbf{netrd} \href{https://github.com/netsiphd/netrd}{[link]} netrd is a multi-purpose library with dozens of state-of-the-art implementations of algorithms for simulating dynamics on networks, measuring the distance between networks, and reconstructing networks from temporal data.

\item \textbf{sunbeam} \href{https://github.com/leotrs/sunbeam}{[link]} sunbeam is a library that uses the non-backtracking matrix to provide functionality for graph mining such as graph distance and graph embedding.

\item \textbf{glee} \href{https://github.com/leotrs/glee}{[link]} glee is a library that uses the simplex geometry of the Laplacian matrix to compute a geometric embedidng of an undirected graph.

\item \textbf{decu} \href{https://github.com/leotrs/decu}{[link]} decu is a suite of command line tools to automate the menial tasks involved in the development of experimental computation projects.

\item \textbf{erdos} \href{http://www.erdosnet.work}{[link]} erdos is an educational site for learning about and practicing Network Science through programming exercises.

\end{itemize}

\end{document}
